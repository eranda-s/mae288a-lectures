\mainmatter%
\setcounter{page}{1}

\lectureseries[\course]{\course}

\auth[\lecAuth]{Lecturer: \lecAuth\\ Scribe: \scribe}
\date{November 3, 2009}

\setaddress%

% the following hack starts the lecture numbering at 11
\setcounter{lecture}{10}
\setcounter{chapter}{10}

\lecture{Controlled Markov Chains II}

\section{Markov Chains Review}
We say that if $N$ is the dimension of the space $\mathbb{X}$ and $N<\infty$ then it means we are working in finite space.
$\pij(u)$ with $u\in\mathcal{U}$ is the probability transition matrix and describes the dynamics of the Markov chain.
The control is $\mu\in\mathcal{M} = \{\mu:\mathbb{X}\to\mathcal{U}\}$.

\begin{align}
\label{eq:pij}
\pij^\mu = \pij(\mu(i))
\end{align}

The cost function and the value function are given by
\begin{align*}
J(i,\mu) &= E\left\lbrace \sum_{t=0}^\infty \alpha^t l(\xi_t,\mu(\xi_t))\right\rbrace \\
V(i) &= \inf_{\mu\in\mathcal{M}}J(i,\mu)
\end{align*}

If $\dim(\mathcal{U})<\infty$, $\dim(\mathcal{M})=\dim(\mathcal{M})<\infty\Rightarrow V(i)=\min_{\mu\in\mathcal{M}}J(i,\mu)$.

The Dynamic Programming Equation (DPE) is found to be
\begin{align}
\label{eq:mcdpe}
V(i) &= \min_{u\in\mathcal{U}}\{l(i,u)+\alpha E[V(\xi_1^\mu)]\} \nonumber \\
&\qquad [\text{find~} \mu \text{~and~} \xi^\mu \text{~by~(\ref{eq:pij})}] \nonumber \\
&= \min_{u\in\mathcal{U}} \left\lbrace l(i,u)+\alpha\sum_{j\in\mathbb{X}}\pij(u)V(j)\right\rbrace
\end{align}

\section{Value Iteration}
Let

\begin{equation*}
\mathcal{G}[\vp](x) \doteq \min_{u\in\mathcal{U}}\left\lbrace l(i,u)+\alpha\sum_{j\in\mathbb{X}}\pij(u)\vp(j)\right\rbrace
\end{equation*}

We solve for $V=\mathcal{G}[V]$ using the following steps:
\begin{enumerate}
\item Guess $V_0$.
\item Find $V_{k+1}=\mathcal{G}[V_k]$.
\item Iterate until convergence criterion is met.
\end{enumerate}

We can use Theorem~\ref{th:bfp} to test if $\mathcal{G}$ is a contraction.
\begin{itemize}
\item Let $||V||_\infty = \max_{i\in\mathbb{X}}|V(i)|$.
\item Let $\mathbb{Y}=\{V:\mathbb{X}\to\mathbb{R}\}$ be the set of real numbers, $\mathbb{R}^N$.
\item Let $V,\tilde{V}\in\mathbb{Y}$.
\end{itemize}
We want to show that $||\mathcal{G}[V]-\mathcal{G}[\tilde{V}]||_\infty\leq\alpha||V-\tilde{V}||_\infty$.
\begin{align*}
||\calg[V]-\calg[\tilde{V}]||_\infty &= \max_{i\in\mathbb{X}}\left[\left|\min_{\uinu}\left\lbrace l(i,u)+\alpha\sum_{j\in\mathbb{X}}\pij(u)V(j)\right\rbrace \right.\right. \\
&\left.\left.\qquad - \min_{v\in\mathcal{U}}\left\lbrace l(i,v)+\alpha\sum_{j\in\mathbb{X}}\pij(v)\tilde{V}(j)\right\rbrace \right|\right] \\
&\leq \max_{i\in\mathbb{X}}\max_{\uinu}\left|\alpha\sum_j\pij(u)[V(j)-\tilde{V}(j)]\right| \\
&\leq \max_{i,u}\alpha\sum_j\pij(u)|V(j)-\tilde{V}(j)| \\
&\leq \alpha\max_{i,u}\sum_j\pij(u)||V-\tilde{V}||_\infty \\
&= \alpha||V-\tilde{V}||_\infty\max_{i,u}\sum_j\pij(u) \\
&= \alpha||V-\tilde{V}||_\infty
\end{align*}
Therefore, $\calg$ is a contraction and by the Banach Fixed Point Theorem (\ref{th:bfp}) the value iteration converges to the unique solution of the DPE\@.
Note that the term $\sum_j\pij(u)=1$ because the probability transition matrix is defined such that all of the probabilities in a row must add up to $1$, so that term need not be considered.

\begin{example}%
\label{ex:valit}
Let $\mathbb{X}=\{1,2,3\}$, $U=\{a,b\}$ and $\alpha=\frac{2}{3}$.
Given
\begin{align*}
\begin{split}
\mathbb{P}(a) = \left[\begin{array}{c c c} \frac{3}{4} & \frac{1}{4} & 0 \\ \frac{1}{2} & \frac{1}{2} & 0 \\ 0 & \frac{3}{4} & \frac{1}{4} \end{array}\right],
\end{split}
\begin{split}
\mathbb{P} (b) = \left[\begin{array}{c c c} \frac{3}{4} & 0 & \frac{1}{4} \\ \frac{1}{4} & \frac{1}{2} & \frac{1}{4} \\ 0 & \frac{1}{2} & \frac{1}{2} \end{array}\right]
\end{split}
\end{align*}
\begin{align*}
l(i,u) = \begin{cases} -2i, & u=a \\ -i, & u=b \end{cases}
\end{align*}
To solve the DPE use (\ref{eq:mcdpe}) to get
\begin{align*}
V(1) &= \min\left\lbrace-2+\frac{2}{3}\left[\frac{3}{4}V(1)+\frac{1}{4}V(2)\right], -1+\frac{2}{3}\left[\frac{3}{4}V(1)+\frac{1}{4}V(3)\right]\right\rbrace \\
V(2) &= \min\left\lbrace-4+\frac{2}{3}\left[\frac{1}{2}V(1)+\frac{1}{2}V(2)\right], -2+\frac{2}{3}\left[\frac{1}{4}V(1)+\frac{1}{4}V(3)\right]\right\rbrace \\
V(3) &= \min\left\lbrace-6+\frac{2}{3}\left[\frac{3}{4}V(2)+\frac{1}{4}V(3)\right], -3+\frac{2}{3}\left[\frac{1}{2}V(2)+\frac{1}{2}V(3)\right]\right\rbrace
\end{align*}
This yields three non-linear equations with three unknowns.
The equations are non-linear because of the $\min$ operator.
To solve for the three unknowns we begin by guessing a solution for $V_0\equiv0$ and getting
\begin{align*}
&V_1(1) = -2, &\qquad &V_1(2) = -4, &\qquad &V_1(3) = -6 \\
&V_2(1) = -\frac{19}{6}, &\qquad &V_2(2) = -6, &\qquad &V_2(3) = -9 \\
&\vdots &\qquad & \vdots &\qquad &\vdots
\end{align*}
This value iteration should continue until the convergence criterion is met.
Recall from \ref{sec:showfcontraction} that the optimal control can be found using these values and typically the optimal control will converge before the value iteration converges.
$\lozenge$
\end{example}

\section{Policy Iteration}
Policy iteration will work for the previous problem calss as well but it is easier here.

The steps for policy iteration are to guess some initial policy (feedback controller) $\muinm$.
With $\xi^\mu$ determined from that guessed policy and $\xi_0=x$ we can find
\begin{align*}
\calg^\mu[\vp](x) &= l(x,\mu(x))+\alpha E[\vp(\xi^\mu)] \\
&= l(x,\mu(x)) + \alpha\sum_{j\in\mathbb{X}}\pxj^\mu\vp(j)
\end{align*}
Note that the term $l(x,\mu(x))$ is the same as the running cost from value iteration.
Given $\mu^k$, let $V^k$ satisfy $V=\calg^{\mu^k}[V]$.
This just means that we are specifying a policy for the contraction mapping.
So, by the above but with $\mathcal{M}=\{\mu^k\}$ we find that $\calg^{\mu^k}$ is a contraction.
This means that there exists a unique solution $V^k$.

The policy iteration is then
\begin{itemize}
\item Given $\mu^k$, let $V^k$ be the unique solution of $V=\calg^{\mu^k}[V]$.
\item Let
\begin{align}
\label{eq:mukplus1}
\mu^{k+1}(x)\in\arg\min_{\uinu}\{l(x,u)+\alpha E[V^k(\xi_1)]\}
\end{align}
with $\xi_0=x$.
\item Go back and forth for policy $\to$ solution $\to$ policy $\to$ solution $\to$ $\ldots$.
\end{itemize}

Suppose $\mu^{\bar{k}+1}=\mu^{\bar{k}}$ so the policy repeats.
Then $V^{\bar{k}+1}$ is the solution of
\begin{align}
\label{eq:politrepeat}
V=\calg^{\mu^{\bar{k}+1}}[V] \therefore V=\calg^{\mu^{\bar{k}}}
\end{align}
But $V^{\bar{k}}$ is the unique solution of (\ref{eq:politrepeat}) so we must have $V^{\bar{k}+1}=V^{\bar{k}}$ and

\begin{equation*}
\therefore \mu^{\bar{k}+2}=\mu^{\bar{k}+1} \Rightarrow V^{\bar{k}+2}=V^{\bar{k}+1} \cdots
\end{equation*}

We also find that

\begin{equation*}
V^{\bar{k}} = V^{\bar{k}+1} = \calg^{\mu^{\bar{k}+1}}[V^{\bar{k}+1}] = \calg^{\mu^{\bar{k}+1}}[V^{\bar{k}}] = \calg[V^{\bar{k}}]
\end{equation*}

by the choice of $\mu^{\bar{k}+1}$.
This is the same as (\ref{eq:mukplus1}) so $V^{\bar{k}}$ must be the unique solution of the DPE $V=\calg[V]$.

\begin{theorem}%
\label{th:dpeuniquesol}
Suppose $\mu^{\bar{k}+1}=\mu^{\bar{k}}$, then $\mu^{\bar{k}+l}=\mu^{\bar{k}}~\forall l\in\mathbb{N}$ and $V^{\bar{k}+l}=V^{\bar{k}}~\forall l\in\mathbb{N}$ and $V^{\bar{k}}$ is the unique solution of the DPE therefore it is the value function.
\end{theorem}

\begin{definition}
For $\muinm$, define $\mathcal{L}^\mu[\vp](x)\doteq \vp(x)-\alpha E[\vp(\xi_1^\mu)]$ with $\xi_0^\mu=x$.
\end{definition}

Note that for $\beta\in\mathbb{R}$ we have

\begin{equation*}
\mathcal{L}^\mu[\beta\vp](x) = \beta\vp(x)-\alpha E[\beta\vp(\xi_1^\mu)] = \beta\mathcal{L}^\mu[\vp](x)
\end{equation*}

and
\begin{align*}
\mathcal{L}^\mu[\vp+\psi](x) &= \vp(x)+\psi(x) - \alpha E[\vp(\xi_1^\mu)+\psi(\xi_1^\mu)] \\
&= \vp(x)-\alpha E[\vp(\xi_1^\mu)]+\psi(x)-\alpha E[\psi(\xi_1^\mu)] \\
&= \mathcal{L}^\mu[\vp](x) + \mathcal{L}^\mu[\psi](x) \\
&\therefore \mathcal{L}^\mu \text{~is~a~linear~operator}
\end{align*}

Let $\hat{l}^\mu(i)\doteq l(i,\mu(i))$, then $V=\calg^\mu[V]$ becomes $V(x)=l(x,\mu(x))+\alpha E[V(\xi_1^\mu)]$ can be rearranged to yield $V(x)-\alpha E[V(\xi_1^\mu)] = l(x,\mu(x))$ and leads to the expression

\begin{equation*}
\mathcal{L}^\mu[V](x) = \hat{l}^\mu(x) \Rightarrow \mathcal{L}^\mu[V] = \hat{l}^\mu
\end{equation*}

This turns the non-linear problem involving the Banach Fixed Point Theorem~\ref{th:bfp} into a linear problem.

In the finite Markov chain case we get

\begin{equation*}
\mathcal{L}^\mu[V](x) = V(x)-\alpha\sum_{j\in\mathbb{X}}\pxj^\mu V(j)
\end{equation*}

Let $V$ be in vector form such that
\begin{align*}
V\equiv\left[\begin{array}{c} V(1) \\ V(2) \\ V(3) \end{array}\right]
\qquad \hat{l}^\mu\equiv~\left[\begin{array}{c} \hat{l}^\mu(1) \\ \hat{l}^\mu(2) \\ \hat{l}^\mu(3) \end{array}\right]
\end{align*}
then $\mathcal{L}^\mu[V]=\hat{l}^\mu$ becomes

\begin{equation*}
\left[I-\alpha\mathbb{P}^\mu\right]V=\hat{l}^\mu
\end{equation*}

which is a finite dimensional linear equality.
We can solve this with linear solvers such as the inverse or pseudo-inverse depending on the specific problem.
Nevertheless, we could use the following to solve $V=\calg^\mu[V]$.
Let $V_0^{k+1}\doteq V^k$.
Then

\begin{equation*}
V_{l+1}^{k+1}=\calg^{\mu^{k+1}}[V_l^{k+1}]
\end{equation*}

By Theorem~\ref{th:bfp} we get $V_l^{k+1}\to V^{k+1}$ is the unique solution of

\begin{equation*}
V=\calg^{\mu^{k+1}}[V]
\end{equation*}

Note that
\begin{align}
\label{eq:polpayoff}
V_1^{k+1} &= \calg^{\mu^{k+1}}[V_0^{k+1}] = \calg^{\mu^{k+1}}[V^k] = \calg[V^k] \nonumber \\
&\qquad[\text{by the choice of } \mu^{k+1}] \nonumber \\
&\leq \calg^{\mu^k}[V^k] = V^k = V_0^{k+1} \nonumber \\
\Rightarrow V_1^{k+1} &\leq V_0^{k+1}
\end{align}
This shows that we get a better payoff at the next iteration using policy iteration.

\begin{lemma}%
\label{lem:contractioninequality}
Suppose $\vp\leq\psi \Rightarrow \vp(x)\leq\psi(x)~\forall x\in\mathbb{X}$, then $\calg^\mu[\vp]\leq\calg^\mu[\psi]$.
\end{lemma}

Suppose $V_{l+1}^{k+1}\leq V_l^{k+1}$ (which is true at $l=0$ by (\ref{eq:polpayoff})), then by Lemma~\ref{lem:contractioninequality} we have that

\begin{equation*}
\calg^{\mu^{k+1}}[V_{l+1}^{k+1}]\leq\calg^{\mu^{k+1}}[V_l^{k+1}] \Rightarrow V_{l+2}^{k+1}\leq V_{l+1}^{k+1}
\end{equation*}

Therefore, by induction $V_l^{k+1}$ is monotonically decreasing in $l$.
Further, we get convergence by Theorem~\ref{th:bfp} and
\begin{align}
\label{eq:vmonotonic}
V_l^{k+1}\to V^{k+1} \therefore V^{k+1}\leq V^k
\end{align}

\begin{theorem}
Suppose there exists $k,l$ such that $\mu^{k+l} = \mu^k$.
Then
\begin{align}
\label{eq:muconverge}
\mu^{k+j}=\mu^k~\forall j\geq 0
\end{align}
and $V^{k+j}=V^k~\forall j\geq0$.
\end{theorem}

\begin{proof}
If $\mu^{k+l}=\mu^k \therefore V^{k+1}=V^k$ since they solve $V=\calg^{\mu^{k+l}}[V]$.
So, by (\ref{eq:vmonotonic})
\begin{align*}
V^{k+j} &= V^k~\forall j\in \interval[open]{0}{l} \\
\Rightarrow \mu^{k+j+1} &= \mu^{k+1} \forall j\in [0,l] \\
&= \mu^k \text{~by~(\ref{eq:muconverge})}
\end{align*}
Then, by Theorem~\ref{th:dpeuniquesol} the proof is done.
\end{proof}
This shows that we can \textit{never} get into a loop.

\begin{corollary}
If $\dim(\mathcal{M})<\infty$ then policy iteration converges exactly in less than $\dim(\mathcal{M})$ steps.
\end{corollary}

Remember that value iteration converges within $\epsilon$, not exactly, so this is a nice feature that is available with policy iteration.

\begin{example}
Start by using the probability transition matrices from Example~\ref{ex:valit}.
Let

\begin{equation*}
l(i,u) = \begin{cases} -2i, & u=a \\ -i, & u=b \end{cases}
\end{equation*}

and guess

\begin{equation*}
\mu^0(x) = \begin{cases} a, & x=2 \\ b, & x=1,3 \end{cases}
\end{equation*}

Then we get

\begin{equation*}
\mathcal{L}^{\mu^0}[V]=\hat{l}^{\mu^0} \to V^0
\end{equation*}

by

\begin{equation*}
[I-\alpha\mathbb{P}^{\mu^0}]V=\hat{l}^{\mu^0}
\end{equation*}

Writing this out yields
\begin{align}
\left[\begin{array}{c c c}
1-\frac{2}{4} & 0 & -\frac{2}{12} \\
\frac{1}{3} & 1-\frac{1}{3} & 0 \\
0 & -\frac{1}{3} & 1-\frac{1}{3}
\end{array}\right] V^0 =
\left[\begin{array}{c}
-1 \\ -4 \\ -3
\end{array}\right]
\end{align}
We can then solve for $V^0$ and find the optimal controller as

\begin{equation*}
\mu(x) = \arg\min\left\lbrace l(x,u)+\alpha\sum\pxj(u)V^0(x)\right\rbrace
\end{equation*}

$\lozenge$
\end{example}
