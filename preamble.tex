% Filename: pcmslmod.tex v.1.4

% Contains modifications to pcms-l.cls (version 1.2d, 1997/01/02)

% Written by David R. Morrison, 1997/04/20; revised 1997/11/23; 1998/2/9;
% 1998/3/13; 1998/3/31; 1998/5/29
% Further modified by Sameer Agarwal 2004/04/05
\typeout{pcmslmod.tex v.1.4, 1998/5/29}

% Summary of changes
%
% 1. A new command \lectureseries, which specifies the title of the lecture
% series and does the page setup for the first page of the series
% (eliminating the need for a \chapter*{} command).  An optional argument
% allows the specification of a shortened title for running heads.
%
% 2. A modification of the \auth command, allowing the specification of a
% shortened author name for running heads.
%
% 3. A modification of the \lecture command, giving it an argument for the
% specification of the lecture name (rather than a separate \lecturename
% command), and incorporating the \chapter*{} command so that the latter
% does not need to be invoked by the user.  From the user's perspective,
% \lecture should function in a very similar fashion to \chapter.
%
% 3A. In v.1.1 of this file, an alternate "starred" form \lecture* is
% introduced, which allows for the inclusion of unnumbered lectures (such
% as a preface, or a list of problems).  The title of this unnumbered
% lecture is flush left if the lecture is the first one (i.e., the
% preface or introduction), otherwise it is set flush right.
%
% 4. Two style corrections in running heads: the body of the running heads
% should be set in medium weight rather than bold, and the author and title
% should be separated by a comma rather than a period.
%
% 5. Modifications of bibliography and index commands, so that their
% headings and running heads have the same style as lectures.  A new
% user-specifiable option \ifBibliographyIsASection (with default value
% \BibliographyIsASectionfalse) is introduced: it should be set to true if
% a user wants separate bibliography sections at the end of each lecture
% rather than a single bibliography at the end.
%
% 6. A minor style change: since the footnote giving the author address is
% typically several lines long, with subsequent lines giving information
% such as e-mail or current addresses which has equal logical weight to the
% address itself, having the first line of the footnote indented presents a
% strange appearance.  As an easy fix, indentation from footnotes was
% removed.
%
% 6A. In v.1.1 of this file, the commands \thanks, \subjclass, \keywords,
% and \date were all implemented; as in most AMS styles, they produce
% additional footnotes on the first page.
%
% 7. The sample files have also been rewritten in a way compatible with these
% changes.
%
% 8. Changes in v 1.4: fixed font size of author, and fixed spacing after
% lecture

\makeatletter

% First, we define a new command \lectureseries, replacing the
% \chapter*{title} command at the head of the file.  An optional argument
% allows a shortened form to be specified for use in running heads.  (The
% thing used in running heads is \thelectureseries -- this is unchanged.)
%
% NB: \part* and \pauth commands could still be used to generate a separate
% title page for an individual lecture series, if that is desired.
%
% We also introduce \iffirstlecture, which is set to true by the
% \lectureseries command so that the top-of-page formatting is not repeated
% by the \lecture command in this case.
\newif\iffirstlecture\firstlecturefalse

\newcommand{\lectureseries}{\firstlecturetrue
              \secdef\@lectureseries\@slectureseries}

\newcommand{\@lectureseries}[2][default]{\chapter*{#2}%
              \gdef\thelectureseries{#1}}

\newcommand{\@slectureseries}[1]{\chapter*{#1}}

% Next, we redefine \auth to allow for the specification of a shortened
% author name in running heads, as an optional argument.  (TeX-nical note:
% it might have been better to write this command and the previous one
% using \@dblarg rather than \secdef.)
\renewcommand{\auth}{\secdef\@auth\@sauth}

\newcommand{\@auth}[2][default]{\vspace{-1pc}{\raggedleft
        \Large\bf\noindent
        #2\endgraf
        \vspace*{2pc}
        }
        \def\@author{#1}%
}

\newcommand{\@sauth}[1]{\vspace{-1pc}{\raggedleft
        \Large\bf\noindent
        #1\endgraf
        \vspace*{2pc}
        }
        \def\@author{#1}%
}

% Next, we redefine \lecture so that \chapter*{} is not needed.  To make
% this work, we only want to insert \chapter*{} starting with the second
% \lecture command, which was why we introduced \iffirstlecture.
%
% Also, \lecture now takes an argument specifying the title, in place of
% the old \lecturename command.

\def\lecture#1{\global\Lecturetrue\global\Monographfalse
\iffirstlecture\else\chapter*{}\fi\firstlecturefalse
  \global\let\sectionmark\@gobble % \lecturemark will be used instead
  \addtocounter{lecture}1\relax
  \refstepcounter{chapter}%
%  \addtocounter{chapter}1\relax % this is done for section numbering
\gdef\thelecturename{#1\unskip}
  {\Large\bfseries
   \raggedleft
   \@xp\uppercase\@xp{\thelecturelabel} {\LARGE\thelecturenum}\\
   \vspace*{3pt}%
   \thelecturename
   \endgraf}%
  \let\@secnumber=\thelecturenum
  \@xp\lecturemark\@xp{\thelecturename}%
  \addcontentsline{toc}{chapter}{%
    \thelecturelabel\ \thelecturenum.\ \thelecturename}%
  \vspace{34\p@}\noindent}

% In v.1.1, \lecture is redefined again, to implement the inclusion of a
% "starred" form \lecture*.

\def\lecture{\global\Lecturetrue\global\Monographfalse
\iffirstlecture\else\chapter*{}\fi%
  \global\let\sectionmark\@gobble % \lecturemark will be used instead
\secdef\@lecture\@slecture}

\def\@lecture[#1]#2{%
  \addtocounter{lecture}1\relax
  \refstepcounter{chapter}%
%  \addtocounter{chapter}1\relax % this is done for section numbering
\gdef\thelecturename{#1\unskip}\firstlecturefalse
  {\Large\bfseries
   \raggedleft
   \@xp\uppercase\@xp{\thelecturelabel} {\LARGE\thelecturenum}\\
   \vspace*{3pt}%
%   \thelecturename
    #2\unskip
   \endgraf}%
  \let\@secnumber=\thelecturenum
  \@xp\lecturemark\@xp{\thelecturename}%
  \addcontentsline{toc}{chapter}{%
%    \thelecturelabel\ \thelecturenum.\ \thelecturename}%
    \thelecturelabel\ \thelecturenum.\ #2}%
  \vspace{34\p@}\noindent}

\def\slecturerunhead#1#2#3{%
    \let\@tempa\chaptername
    \uppercasenonmath{\@tempa}%
    \def\@tempb{#3\unskip}%
    \uppercasenonmath{\@tempb}%
    {\normalfont\@tempb}
    }
\def\slecturemark{%\let\@secnumber\@empty
%    \@secmark\markright\sectionrunhead\sectionname}%
    \@secmark\markright\slecturerunhead\chaptername}%


\def\@slecture#1{%
\iffirstlecture
%  \addtocounter{lecture}1\relax
%  \refstepcounter{chapter}%
%%  \addtocounter{chapter}1\relax % this is done for section numbering
\gdef\thelecturename{#1\unskip}\firstlecturefalse
  {\Large\bfseries
%   \raggedleft
%   \@xp\uppercase\@xp{\thelecturelabel} {\LARGE\thelecturenum}\\
%   \vspace*{3pt}%
%\noindent\@xp\uppercase\@xp{\thelecturename}
\noindent\thelecturename
   \endgraf}%
  \let\@secnumber=\thelecturenum
  \@xp\slecturemark\@xp{\thelecturename}%
%\markright\thelecturename
  \addcontentsline{toc}{chapter}{%
    \thelecturename}%
 \vspace{-6\p@}\noindent
%\noindent
\else
%  \addtocounter{lecture}1\relax
%  \refstepcounter{chapter}%
%%  \addtocounter{chapter}1\relax % this is done for section numbering
\gdef\thelecturename{#1\unskip}\firstlecturefalse
  {\Large\bfseries
   \raggedleft
%   \@xp\uppercase\@xp{\thelecturelabel} {\LARGE\thelecturenum}\\
%   \vspace*{3pt}%
   \@xp\uppercase\@xp{\thelecturename}
   \endgraf}%
  \let\@secnumber=\thelecturenum
  \@xp\slecturemark\@xp{\thelecturename}%
%\markright\thelecturename
  \addcontentsline{toc}{chapter}{%
    \thelecturename}%
  \vspace{34\p@}\noindent
\fi}

% We make the following changes to definitions of running heads:
% (1) add \textmd so that the header is not boldface
% (2) use a comma, not a period, to separate author and lectureseries

\ifLecture
  \def\chapterrunhead#1#2#3{%
    \let\@tempa\@author
    \uppercasenonmath{\@tempa}%
    \uppercasenonmath{\thelectureseries}%
    \textmd{\@tempa, \thelectureseries}
    }
  \def\lecturerunhead#1#2#3{%
    \let\@tempa\chaptername
    \uppercasenonmath{\@tempa}%
    \def\@tempb{#3\unskip}%
    \uppercasenonmath{\@tempb}%
    \textmd{\@tempa\ #2. \@tempb}
    }
\else
  \let\chapterrunhead\partrunhead
\fi

% For the bibliography, we do two things
% (1) we introduce \ifBibliographyIsASection (default is false) to decide
% if a section or a chapter.  When its a chapter, but we are in lecture
% mode, we use the lecture style of headings.  If its a section, it should
% NOT be in backmatter.
% (2) we fix the running heads to be consistent with everything else.

\newif\ifBibliographyIsASection\BibliographyIsASectionfalse

  \def\bibliomark{%\let\@secnumber\@empty
%    \@secmark\markright\sectionrunhead\sectionname}%
    \@secmark\markright\bibliorunhead\chaptername}%

  \def\bibliorunhead#1#2#3{%
    \let\@tempa\chaptername
    \uppercasenonmath{\@tempa}%
    \def\@tempb{#3\unskip}%
    \uppercasenonmath{\@tempb}%
    \textmd{\@tempb}
    }

\def\thebibliography#1{%
  \ifBibliographyIsASection
    \section*\refname
    \if@backmatter
      \markboth{\refname}{\refname}%
    \fi
  \else
\chapter*{}
  {\Large\bfseries
   \raggedleft
   \@xp\uppercase\@xp{\bibname} \\
   \endgraf}%
  \let\@secnumber=\thelecturenum
  \@xp\bibliomark\@xp{\bibname}%
  \addcontentsline{toc}{chapter}{%
    \bibname}%
  \vspace{34\p@}\noindent
  \fi
  \normalsize\labelsep .5em\relax
  \list{\@arabic\c@enumi.}{\settowidth\labelwidth{\@biblabel{#1}}%
  \leftmargin\labelwidth
  \advance\leftmargin\labelsep
%	\bibsetup\relax
	\usecounter{enumi}}\sloppy
  \clubpenalty9999 \widowpenalty\clubpenalty  \sfcode`\.\@m}

% We also want to change the headings and running heads for the index.  We
% only do this in the case of a lecture (so the previous definition will still
% be invoked in the case of a monograph.)

  \def\indexmark{%\let\@secnumber\@empty
%    \@secmark\markright\sectionrunhead\sectionname}%
    \@secmark\markright\indexrunhead\chaptername}%

  \def\indexrunhead#1#2#3{%
    \let\@tempa\chaptername
    \uppercasenonmath{\@tempa}%
    \def\@tempb{#3\unskip}%
    \uppercasenonmath{\@tempb}%
    \textmd{\@tempb}
    }

\ifLecture
\def\theindex{\cleardoublepage
\@restonecoltrue\if@twocolumn\@restonecolfalse\fi
\columnseprule \z@ \columnsep 35pt
\def\indexchap{\@startsection
		{chapter}{1}{\z@}{8pc}{34pt}%
		{\raggedleft
		\Large\bfseries}}%
 \twocolumn[\indexchap[{\indexname}]{\@xp\uppercase\@xp{\indexname}}]
%		\Large\bfseries}}%
% \twocolumn[\indexchap*{\@xp\uppercase\@xp{\indexname}}]
% \@mkboth{{\indexname}}{{\indexname}}%
  \@xp\indexmark\@xp{\indexname}%
	\thispagestyle{plain}\let\item\@idxitem\parindent\z@
	 \footnotesize\parskip\z@ plus .3pt\relax\let\item\@idxitem}
\fi

% Finally, a small stylistic change: for the footnote giving the author
% address, indenting the footnote doesn't look good (in my opinion) due to
% the email line NOT being indented. So we change:
%
% \def\@makefntext{\indent\@makefnmark}
%
% to

\def\@makefntext{\noindent\@makefnmark}

% In v.1.1, we also implement \thanks and other commands which make
% first-page footnotes:

\def\setaddress{%
  {\let\@makefnmark\relax  \let\@thefnmark\relax
        \nobreak
        \addressnum@=\z@
        \loop\ifnum\addressnum@<\addresscount@\advance\addressnum@\@ne
           \footnote{$^{\hbox{\tiny\number\addressnum@}}$%
           \csname @address\number\addressnum@\endcsname
           \csname @curraddr\number\addressnum@\endcsname
           \csname @email\number\addressnum@\endcsname}\repeat
  \ifx\@empty\@date\else \@footnotetext{\@setdate}\fi
  \ifx\@empty\@subjclass\else \@footnotetext{\@setsubjclass}\fi
  \ifx\@empty\@keywords\else \@footnotetext{\@setkeywords}\fi
  \ifx\@empty\thankses\else \@footnotetext{%
    \def\par{\let\par\@par}\@setthanks}\fi
    }%
%  \@setcopyright
}

% fix blank pages (Dan Freed -- November 25, 1997)

\def\@tmpevenhead{\relax}

\def\cleardoublepage{\clearpage\if@twoside \ifodd\c@page\else
    \let\@tmpevenhead\@evenhead \let\@evenhead\relax\hbox{}\eject
    \let\@evenhead\@tmpevenhead\if@twocolumn\hbox{}\newpage\fi\fi\fi}

% define \copyrightyear to be \currentyear (Dan Freed -- March 13, 1998)

\def\@setcopyright{%
  \let\copyrightyear\currentyear             % DF
  \insert\copyins{\hsize\textwidth
    \parfillskip\z@ \leftskip\z@\@plus.9\textwidth
    \fontsize{6}{7\p@}\normalfont\upshape
    \everypar{}%
    \vskip-\skip\copyins \nointerlineskip
    \noindent\vrule\@width\z@\@height\skip\copyins
    \copyright\copyrightyear\ American Mathematical Society\par
    \kern\z@}%
}

% macro to put in blank page at beginning for 2-up viewing

\def\BlankPage{\pagestyle{empty}\thispagestyle{empty}\null\vfil\eject}

%fix font size of author:

\renewcommand{\@auth}[2][default]{{\raggedleft
        \begingroup
  \fontsize{\@xivpt}{18}\bfseries%\centering
  #2\par \endgroup
        \vspace*{2pc}
        }
        \def\@author{#1}%
}

\renewcommand{\@sauth}[1]{{\raggedleft
        \begingroup
  \fontsize{\@xivpt}{18}\bfseries%\centering
  #1\par \endgroup
        \vspace*{2pc}
        }
        \def\@author{#1}%
}

%fix spacing after lecture

\def\@lecture[#1]#2{%
  \addtocounter{lecture}1\relax
  \refstepcounter{chapter}%
\gdef\thelecturename{#1\unskip}\firstlecturefalse
  {\Large\bfseries
   \raggedleft
   \@xp\uppercase\@xp{\thelecturelabel} {\LARGE\thelecturenum}\\
   \vspace*{3pt}%
    #2\unskip
   \endgraf}%
  \let\@secnumber=\thelecturenum
  \@xp\lecturemark\@xp{\thelecturename}%
  \addcontentsline{toc}{chapter}{%
    \thelecturelabel\ \thelecturenum.\ #2}%
  \vspace{10\p@}\noindent}

\def\@slecture#1{%
\iffirstlecture
\gdef\thelecturename{#1\unskip}\firstlecturefalse
  {\Large\bfseries
\noindent\thelecturename
   \endgraf}%
  \let\@secnumber=\thelecturenum
  \@xp\slecturemark\@xp{\thelecturename}%
  \addcontentsline{toc}{chapter}{%
    \thelecturename}%
 \vspace{-6\p@}\noindent
\else
\gdef\thelecturename{#1\unskip}\firstlecturefalse
  {\Large\bfseries
   \raggedleft
   \@xp\uppercase\@xp{\thelecturename}
   \endgraf}%
  \let\@secnumber=\thelecturenum
  \@xp\slecturemark\@xp{\thelecturename}%
  \addcontentsline{toc}{chapter}{%
    \thelecturename}%
  \vspace{10\p@}\noindent
\fi}

% End of modifications to pcms-l.cls.
\makeatother

% Packages.
\usepackage{amsmath,amsfonts,amsthm,amssymb}
\usepackage{setspace}
\usepackage{fancyhdr}
\usepackage{lastpage}
\usepackage{extramarks}
\usepackage{chngpage}
\usepackage{soul}
\usepackage[usenames,dvipsnames]{color}
\usepackage{graphicx,float,wrapfig}
\usepackage{subfig}
\usepackage{ifthen}
\usepackage{listings}
\usepackage{courier}
\usepackage[bw,framed,numbered]{mcode}
\usepackage{microtype}
\usepackage{appendix}
\usepackage{hyperref}
\usepackage[all]{hypcap}
\usepackage{url}
\usepackage{wrapfig}
\addtolength{\textwidth}{1cm}

% EQUATION NUMBERING AND THEOREM SETUP
\numberwithin{section}{chapter}
\numberwithin{equation}{chapter}
\theoremstyle{plain}
\newtheorem{theorem}[equation]{Theorem}
\newtheorem{lemma}[equation]{Lemma}
\theoremstyle{definition}
\newtheorem{definition}[equation]{Definition}
\newtheorem{exercise}{Exercise}
\newtheorem{problem}{Problem}
\newtheorem*{remark}{Remark}

% Set enumerate to use letters, not numbers for problem parts.
\renewcommand{\theenumi}{\alph{enumi}}
\renewcommand{\labelenumi}{(\theenumi)}

% AUTHOR-DEFINED MACROS:
%\renewcommand{\v}[1]{\mathbf{#1}}
\renewcommand{\v}[1]{\boldsymbol{#1}}
